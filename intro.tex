\section{Introduction, motivations}
Ce cours a pour objectif d'introduire plusieurs outils mathématiques nécessaire en géo-physique. Les concepts développés permettront d'aborder par exemple la mécanique des fluides, l'élasticité, le géomagnétisme ou la sismologie. De façon plus général, ce cours donne les outils indispensables à la manipulation de champs physiques en milieux anisotropes.
L'aspect purement mathématique, tel que les démonstrations formelles ou l'introduction de nouvelles algèbres n'est pas l'objectif de ce cours. Il se contente de donner les clés de problèmes physiques récurrents dans les disciplines nommées plus haut. Ainsi, suite à la lecture de ce cours, l'étudiant ne devrait pas être bloqué par la mise en équation de problèmes physiques et géo-physiques impliquant un champ sur la Terre, sans pour autant avoir une compréhension très précise des objets mathématiques manipulés. 
Pour résumer, ce cours de mathématiques n'est pas une fin en soit, il constitue simplement un outils pour surmonter les difficultés théoriques à l'étude du système Terre, et pourrait être vu comme une caisse à outils.
