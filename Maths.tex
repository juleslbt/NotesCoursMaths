\documentclass[12pt,fleqn]{book}
\usepackage{indentfirst,amsmath}
\usepackage{amsfonts,amssymb}


\usepackage[french]{babel}
\usepackage[utf8]{inputenc}

\usepackage[bookmarks=true,
bookmarksnumbered = true,
hyperindex,
backref=true,
colorlinks]{hyperref}

\usepackage{fancyheadings}
\pagestyle{fancyplain}
\addtolength{\headheight}{2.5pt} % si 12pt
\renewcommand{\chaptermark}[1]{\markboth{#1}{#1}}
\renewcommand{\sectionmark}[1]{\markright{\thesection\ #1}}
\lhead[\fancyplain{}{\bfseries\thepage}]{\fancyplain{}{\bfseries\rightmark}}
\rhead[\fancyplain{}{\bfseries\leftmark}]{\fancyplain{}{\bfseries\thepage}}
\cfoot{}

\begin{document}

\title{Mathématiques L3}
\author{Les étudiants de la licence}
\maketitle

\chapter{Les fonctions de plusieurs vriables}
\label{cha:les-fonctions-de}

\chapter{Les opérateurs de dérivation}
\label{cha:les-operateurs-de}

\chapter{Équations fondamentales de la physique}
\label{cha:equat-fond-de}

\chapter{Harmoniques sphériques}
\label{cha:harm-spher}

\chapter{Introduction aux tenseurs}
\label{cha:intr-aux-tens}




\end{document} 

